\documentclass{article}


\usepackage{arxiv}

\usepackage[utf8]{inputenc} % allow utf-8 input
\usepackage[T1]{fontenc}    % use 8-bit T1 fonts
\usepackage{hyperref}       % hyperlinks
\usepackage{url}            % simple URL typesetting
\usepackage{booktabs}       % professional-quality tables
\usepackage{amsfonts}       % blackboard math symbols
\usepackage{nicefrac}       % compact symbols for 1/2, etc.
\usepackage{microtype}      % microtypography
\usepackage{lipsum}

\title{A template for the \emph{arxiv} style}

\author{
  Shriram Varadarajan \\
  Department of Elecrical Computer engineering\\
  University of Iowa\\
  Iowa city, IA, 52242 \\
  \texttt{shriram-varadarajan@uiowa.edu} \\
  %% examples of more authors
   \And
 Jacob Nishimura \\
  Department of Elecrical Computer engineering\\
  University of Iowa\\
  Iowa city, IA, 52242 \\
  \texttt{jacob-nishimura@uiowa.edu} \\
\And
Joseph Kim \\
  Department of Biomedical Engineering\\
  University of Iowa\\
  Iowa city, IA, 52242 \\
  \texttt{joseph-kim@uiowa.edu} \\
  %% \AND
  %% Coauthor \\
  %% Affiliation \\
  %% Address \\
  %% \texttt{email} \\
  %% \And
  %% Coauthor \\
  %% Affiliation \\
  %% Address \\
  %% \texttt{email} \\
  %% \And
  %% Coauthor \\
  %% Affiliation \\
  %% Address \\
  %% \texttt{email} \\
}


\begin{document}
\maketitle

\begin{abstract}
This is our research paper 
\end{abstract}


% keywords can be removed
\keywords{First keyword \and Second keyword \and More}


\section{Introduction}
\begin{itemize}
\item Overview of the problem and the relevant background knowledge must be described in a section titled "Introduction". 
\item The introduction section should be around 1-1.5 pages, and certainly no longer than 2 pages using the submission template above.
\item Your audience should be fellow graduate students/faculty in a different discipline. Hence, you must educate them with an easy-to-understand language, so that the readers are ready to digest the remainder of your project description. You can assume the audience has a basic background knowledge on the topics such as AI/ML/DL. 
\item Related works should be referenced, so that the readers can have some historical context ("what other people did/do"). However, the introduction section should not be too technical or jargony.
\end{itemize}


\section{Problem Definition}
\label{sec:pb}
If the intro section was a place for mostly lay-person's description of your project, "Problem Definition" section is where you can use technical terms to define your problem more precisely.\\
Be very precise and explicit about input-output parameters to your machine learning problem. For example, "I will make a machine learning model that predicts house price"  is not a good problem statement. Instead, be more specific about what goes into your model and what will come out of your model. For example, "The model will take a color photograph (RGB image) of a house resized to 224-by-224 alongside other metadata including 'build year,' 'days on market,' 'square footage,' and 'school district,' and predict the dollar amount (normalized in range [0,1]) of the actual market price of the house as an output" is a better way to state your problem. If you have too many parameters to be listed in one sentence, creating a table listing inputs and outputs, as well as their data types (e.g. color image, grayscale image, time-series, scalar, string, ...) would be a great idea.
The problem definition section should be around 0.5 - 1 page, but certainly no more than that.\\

See Section \ref{sec:pb}.

\subsection{Headings: second level}


\subsubsection{Headings: third level}


\paragraph{Paragraph}


\section{Data}
\label{sec:Data}

\begin{itemize}
\item What is available/not available in the data set (in conjunction with your input-output description in Problem Definition)
\item How do they look like? (insert figures showing some data samples)
\item How are they collected? What device/modality/sensor/etc. was used?
\item How are they formatted? What do you need to do to parse them? Is there a parser available, or do you need to build your own?
\item (If human subject data) A statement indicating the IRB status and compliance with other human subject research protocols.
\end{itemize}

\subsection{Figures}
Figures \\
See Figure \ref{fig:fig1}. Here is how you add footnotes. \footnote{Sample of the first footnote.}


\begin{figure}
  \centering
  \fbox{\rule[-.5cm]{4cm}{4cm} \rule[-.5cm]{4cm}{0cm}}
  \caption{Sample figure caption.}
  \label{fig:fig1}
\end{figure}

\subsection{Tables}

See awesome Table~\ref{tab:table}.

\begin{table}
 \caption{Sample table title}
  \centering
  \begin{tabular}{lll}
    \toprule
    \multicolumn{2}{c}{Part}                   \\
    \cmidrule(r){1-2}
    Name     & Description     & Size ($\mu$m) \\
    \midrule
    Dendrite & Input terminal  & $\sim$100     \\
    Axon     & Output terminal & $\sim$10      \\
    Soma     & Cell body       & up to $10^6$  \\
    \bottomrule
  \end{tabular}
  \label{tab:table}
\end{table}




\bibliographystyle{unsrt}  
%\bibliography{references}  %%% Remove comment to use the external .bib file (using bibtex).
%%% and comment out the ``thebibliography'' section.


%%% Comment out this section when you \bibliography{references} is enabled.
\begin{thebibliography}{1}

\bibitem{kour2014real}
George Kour and Raid Saabne.
\newblock Real-time segmentation of on-line handwritten arabic script.
\newblock In {\em Frontiers in Handwriting Recognition (ICFHR), 2014 14th
  International Conference on}, pages 417--422. IEEE, 2014.

\bibitem{kour2014fast}
George Kour and Raid Saabne.
\newblock Fast classification of handwritten on-line arabic characters.
\newblock In {\em Soft Computing and Pattern Recognition (SoCPaR), 2014 6th
  International Conference of}, pages 312--318. IEEE, 2014.

\bibitem{hadash2018estimate}
Guy Hadash, Einat Kermany, Boaz Carmeli, Ofer Lavi, George Kour, and Alon
  Jacovi.
\newblock Estimate and replace: A novel approach to integrating deep neural
  networks with existing applications.
\newblock {\em arXiv preprint arXiv:1804.09028}, 2018.

\end{thebibliography}


\end{document}